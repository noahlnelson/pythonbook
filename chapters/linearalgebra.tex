\chapter{Linear Algebra}
\label{chap:linalg}
\section{Matrices}
Matrix math is different from normal math.  That will become more
clear after you take a linear algebra class.  If you want to do matrix
math or linear algebra, \code{numpy}'s \code{matrix} object is what you
want to use.  A \code{matrix} object can be created in a few ways.  Here are
a few examples

\subsection*{Creating Matrices}
A \texttt{matrix} object can be created in a
few ways.  Here are a few examples
>>>>>>> Added plotting chapters
\begin{Verbatim}
from numpy import matrix

a = matrix('1 2; 3 4')  # Create a 2 x 2 matrix from string
b = matrix([[1,2],[3,4]])  # Create 2 x 2 matrix from list
c = matrix('1;2;3;4')  #Create column vector: a 4 x 1 matrix
\end{Verbatim}
The first definition is a nice way to create a matrix from a string.
The \code{;} indicates the end of the rows.  You can also convert a
list or array into a matrix.  Note that if you print a matrix object
to screen, it will probably look the same as an array or list (or
similar).  The differences are the things you can do with a matrix
object as compared to an array object, or list object.

\subsection*{Math with Matrices}
Once the matrix is defined, lots of cool and useful math becomes
available to you.  Here are a few examples:

\begin{Verbatim}
from numpy import matrix

a = matrix('1 2; 3 4')  # Create 2 x 2 matrix from string
b = matrix('5 6; 8 9')  # Create 2 x 2 matrix from string
col = matrix('3;4')  # Create 2 x 1 column vector

c = a.T   # Transpose the matrix
d = a.I   # Find inverse of matrix
e = a.H   # Find conjugate transpose of matrix
f = a * b # Matrix multiplication
g = b * col # Multiply matrix b to column vector
h = a**2   # Square the matrix. Not the same as squaring an array.
\end{Verbatim}
