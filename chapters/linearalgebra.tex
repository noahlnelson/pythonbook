\chapter{Linear Algebra}
\label{chap:linalg}
Linear algebra is the study of sets of linear equations and their
solutions.  Matrices and column vectors are common mathematical
objects used in this field of Mathematics.  In this chapter we will
illustrate how to use Python (numpy actually) to solve common problems
encountered in the field of linear algebra.
\section{Matrices}
We have already seen the many mathematical advantages associated with
using arrays.  Numpy also has a matrix object that was specially
designed to perform operations that are typical to matrices.  If you
want to do matrix math or linear algebra, \code{numpy}'s \code{matrix}
object is a good choice.  A \code{matrix} object can be created in a
few ways.  Here are a few examples

\subsection*{Creating Matrices}
A \texttt{matrix} object can be created in a
few ways.  Here are a few examples
\begin{Verbatim}
from numpy import matrix

a = matrix('1 2; 3 4')  # Create a 2 x 2 matrix from string
b = matrix([[1,2],[3,4]])  # Create 2 x 2 matrix from list
c = matrix('1;2;3;4')  #Create column vector: a 4 x 1 matrix
\end{Verbatim}
The first definition is a nice way to create a matrix from a string.
The \code{;} indicates the end of the rows.  You can also convert a
list or an array into a matrix.  Note that if you print a matrix object
to screen, it will probably look the same as an array or list (or
similar).  The differences are the things you can do with a matrix
object as compared to an array object, or list object.

\subsection*{Math with Matrices}
Once the matrix is defined, lots of cool and useful math becomes
available to you.  Here are a few examples:

\begin{Verbatim}
from numpy import matrix

a = matrix('1 2; 3 4')  # Create 2 x 2 matrix from string
b = matrix('5 6; 8 9')  # Create 2 x 2 matrix from string
col = matrix('3;4')  # Create 2 x 1 column vector

c = a.T   # Transpose the matrix
d = a.I   # Find inverse of matrix
e = a.H   # Find conjugate transpose of matrix
f = a * b # Matrix multiplication
g = b * col # Multiply matrix b to column vector
h = a**2   # Square the matrix. Not the same as squaring an array.
\end{Verbatim}
 Many other useful functions are available to matrix objects.  Online
 documentation is very helpful.  A small sampling of what is available
 is found in Table \ref{tab:matrixTable}

\section{Solving a Set of Linear Equations}
A common problem encountered in Linear Algebra is that of solving a
set of linear equations.  Here is an example of a set of two linear
equations, with two unknown variables:

\begin{equation}
3 x + y = 9\qquad
x + 2 y = 8
\end{equation}
This problem can be represented
in matrix form like this:

\begin{equation}
\mathbf{A}.\vec{x} = \vec{b}
\end{equation}
where
\begin{equation}
\mathbf{A} = \left( \begin{tabular}{cc}
3 & 1 \\
1 & 2 \\
\end{tabular}
\right)
\end{equation},
\begin{equation}
\vec{b} = \left( \begin{tabular}{c}
9  \\
8  \\
\end{tabular}
\right)
\end{equation}, 
and
\begin{equation}
\vec{x} = \left( \begin{tabular}{c}
x  \\
y  \\
\end{tabular}
\right)
\end{equation}, 
\marginpar{\footnotesize\captionsetup{type=table} \vspace{-2.5in}
\begin{tabular}{lp{1.05in}}
\code{a.trace()}        & Calculate trace of matrix a.\\
\code{a.shape}        & Returns the shape of matrix a.\\
\code{a.H}        & Returns the conjugate transpose of matrix a.\\
\code{a.diagonal()}        & Return diagonal elements of matrix a.\\
\code{(a==b).all()}        & Check if all elements of a matrix satisfy
a condition.\\
\code{(a==b).any()}        & Check if any elements of a matrix satisfy
a condition.\\
\end{tabular}
\captionof{table}{A sampling of functions that work on matrix objects in
  numpy.  See online documentation for further details.\label{tab:matrixTable}}
}

Numpy has a function called \texttt{solve} that will solve this
problem.  This is how it is used:
\begin{Verbatim}
from numpy.linalg import solve
from numpy import matrix

a = matrix('3,1;1,2')
b = matrix('9;8')
x = solve(a,b)
\end{Verbatim}
Please note that $a$ and $b$ need not be matrix objects for the
\code{solve} function to work correctly.  Array objects will work just
fine.  However, if you do use matrix objects, you could also solve the
problem like this:
\begin{Verbatim}
x = a.I * b # Multiply the inverse of A by b.
\end{Verbatim}


\section{Eigenvalue Problems}
Eigenvalue problems show up in many branches of science
including:Quantum Mechanics, signal processing, and Geology. The
eigenvalue problem is similar to the linear system of equations
problem except that the r.h.s is not known.  Here is the mathematical
statement for an eigenvalue problem.

\begin{equation}
\vec{A}.\vec{x} = \lambda \vec{x}
\end{equation}
The solution to this problem yields a value for $\lambda$ and a
corresponding vector $\vec{x}$ corresponding to it.  Here is how the
problem would be solved in Python:
\begin{Verbatim}
from numpy.linalg import eig
from numpy import matrix

a = matrix('1,-1;1,1')
vals,vecs = eig(a)
\end{Verbatim}
Notice that the function \code{eig} returns two things: i) the
eigenvalues ($\lambda$) and ii) the eigenvectors.  Since it returns
two things, I need two variables on the l.h.s to save them to.  The
first thing is the eigenvalues and the second is the corresponding
eigenvectors, which are the columns of the matrix that is returned.
