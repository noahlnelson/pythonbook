\chapter{Running Python}
\label{chap:RunningPython}

\marginfig{Figures/PythonCW.jpg}{Canopy's environment for running
  Python code.}

Python is a computer programming language (don't freak out) with broad
applicability in science and engineering.  For those that are brand
new to computer programming, Python is simply a way to communicate a
set of instructions to your computer.  You'll quickly learn that your
computer is great at doing exactly what you tell it to do.  If you
find that your computer isn't doing what you think it should, it is
not because the machine is malfunctioning, rather you just probably
don't fully understand what you are telling it to do.

There are two ways to run Python. One is at the command line\sidenote{If you are already comfortable with command line, then you probably don't need any help getting started. Feel free to use whatever editor you'd like. You'll need to make sure that you have Python 3 installed along with the following packages: numpy, scipy, matplotlib, iPython, sympy, and nose. These packages come pre-bundled with the Canopy distribution of Python, so if you installed Canopy, you already have them.} and the
other is through an IDE (integrated development environment). If you
don't know what the command line is, I recommend you start out using
an IDE. A good IDE for Python is called Canopy, which can
be downloaded \href{https://store.enthought.com/downloads/}{here}. This text is written for Python 3, so choose the Python 3.x download that matches your operating system (Windows, Mac, or Linux).


\section{Running Your First Program}

Once you have downloaded and installed Canopy, launch the editor. You should see a window that looks similar to the one displayed in the margin. (You may have to choose "Create a new file".) The Canopy Editor has three main windows: The file browser, the editor itself, and the Python console (Labeled "Python").
The file browser shows your local file directory.  It is nice to have when you write larger Python programs. To keep big programs easy to read, we often break them into several files.
The editor itself will be where we write our Python programs. We'll address the Python Console in a moment. Right now, type

\begin{Verbatim}
print('Hello World')
\end{Verbatim}

into the editor. Save your program as \code{myFirstProgram.py}, then click the green
arrow above the editor. If you look down at the console it should say:

\begin{Verbatim}
%run "where you saved the file/myFirstProgram.py"
Hello World
\end{Verbatim}
(If you don't see the words "Hello World", double check that what you've written in the editor and matches the above code exactly.) Congratulations, you just ran your first Python program. You told the computer to write "Hello World" on the console.
Now try changing this line:
\begin{Verbatim}
print('Hello World')
\end{Verbatim}
to this:
\begin{Verbatim}
print('I can do Python!')
\end{Verbatim}
and run your program again.\sidenote{You only need to do a "save as" on new files. Once you've given it a filename, Canopy will save and run your file every time you click the green arrow button.}
\section{The Python Console}
The Python console serves two purposes: First, it shows our program's outputs. (The "Hello World" from earlier.) Second, it allows us to do interactive Python. Try typing
\begin{Verbatim}
print('Hello World')
\end{Verbatim}
directly into the console, then press enter.  You should see it print
\code{Hello World}, just like it did for your program. The console
will run any Python command that you enter into it. It can be very
useful for short, quick calculations, or for looking at data when you
are trying to figure out what your program is doing. However, it is
impossible to save the commands that you enter into the console, so
you should do most of your programming in the editor.


\section{It's a Calculator}
The very easiest, yet meaningful thing you can with Python is to
perform simple math. Simple math can be performed pretty much just as
you would expect. Here are a few examples\sidenote{The \code{#} marks something called a comment.  Python knows to ignore anything on a line that comes after a \code{#}.  They exist so that you can leave notes to anyone reading the program, without affecting what the program does.}
\begin{Verbatim}
print(1+2)
print(5.0/6.0)
print(5**6)  # 5 raised to the power 6
print(678 * (3.5 + 2.8)**3.)
\end{Verbatim}
Note that typing the calculation alone, without the \code{print}
statement, will not produce any output to the screen.  Even though the
calculation has been performed, you will not see the result unless you
\code{print} it. Throughout this book, the \code{print}
statement will usually be omitted in code examples to maintain
brevity.  You should always print your result so you can see what you
have done.
