\chapter{Plotting}
\label{chap:Plotting}

Creating plots is an important task in science and engineering.  The
old addage ``A picture is worth a thousand words!'' is wrong.... it's
worth way more than that if you do it right.  
\section{Plotting Data}
Python does not plot functions.  This is true of any computer
software used to make plots.  Rather, the computer samples the
function at a series of points in the domain and plots those points.
In python, we can create a plot like this using a library called
\texttt{matplotlib}.  A simple scatter plot can be generated like this

\begin{Verbatim} 
from matplotlib import pyplot  #Import necessary functions

xData = [1.2,2.3,8.3,10.5]  # x-coordinates of the data points
xData = [2.2,8.5,1.2,3.5]   # y-coordinates of the data points
pyplot.scatter(xData,yData)  # make the plot
pyplot.show()                # Display the plot to your screen.
\end{Verbatim} 
Here we have plotted some data and displayed them to the screen.
\subsection*{Customizing Scatter Plots}
You can easily modify the style,color, and size of the plot markers.
The change the style of the plot marker, add the optional
\texttt{marker} argrument to the scatter function, like this
\begin{Verbatim} 
pyplot.scatter(xData,yData,marker='+') # Use plus sign as the marker  
\end{Verbatim} 
A list of plot marker styles is given in table \ref{tab:styles}
\marginpar{\footnotesize\captionsetup{type=table}
  \vspace{0.5in}
\begin{tabular}{ll}
\texttt{'+'} & Plus \\
\texttt{'.'} & Point \\
\texttt{','} & Pixel \\
\texttt{'v'} & Triangle down \\
\texttt{'8'} & Octagon \\
\texttt{'s'} & Square \\
\texttt{'p'} & Pentagon \\
\texttt{'x'} & X \\
\end{tabular}
\captionof{table}{Sampling of available plot markers.\label{tab:styles}}
}


\section{Plotting Functions}

Sometimes you actually do want to plot a function and don't really
want to see the data points individually.  You can do this using the
\texttt{plot} function (also inside of \texttt{matplotlib}).
\begin{Verbatim}
from math import sin
from matplotlib import pyplot
xData = range(0,10):
yData = [sin(x) for x in xData]
pyplot.plot(xData,yData)
pyplot.show()
\end{Verbatim}
The \texttt{plot} function connects the data points being plotted.  In
this case, we only had 10 data points, so the plot looked pretty
jagged.  To get a better plot of the function, we need to generate
more data points in our range (0,10).  Unfortunately, Python's
built-in \texttt{range} function can't do that for us.  However, the
\texttt{numpy} library has a function called \texttt{arange} that does
just that.
\begin{Verbatim}
from math import sin
from numpy import arange
from matplotlib import pyplot
xData = arange(0,10,.01):
yData = [sin(x) for x in xData]  #<-  This is cool, but probably new
                                 #    to you.
pyplot.plot(xData,yData)
pyplot.show()
\end{Verbatim}
Notice that \texttt{arange} takes three arguments: starting position,
ending position, and step size.  You should take a look at the
variable \texttt{xData} to see what you just created.  Also note the
line that was used to generate the function values (\texttt{yData}).
This is a compact way to combine a loop with a list and is worthwhile
to learn.  Notice that I am evaluating \texttt{sin(x)} at each point
in my list called \texttt{xData} and saving each number in a list
(hence the brackets enclosing the whole things).  When
I'm all done, I have a list of function values evaluated at the exact
points in \texttt{xData}.\sidenote{Think about this until it makes
  sense.  It will save you time.}

\section{Histograms}
Histograms are very useful plots when thinking about mean values and
spread in data sets.  As with the other plots, the library
\texttt{matplotlib} has the needed function to create this plot.
Below is an example
\begin{Verbatim}
from matplotlib import pyplot
measurements = [3.46,2.48,1.98,3.42,3.56,3.87,3.9,4.2,2.9]
pyplot.hist(measurements)
pyplot.show()
\end{Verbatim}


\section{Displaying multiple plots}
Frequently you will want to display multiple plots on the same figure.
 To do this, simply execute multiple plot commands and wait until the
 end to use the \texttt{show} command, like this:
\begin{Verbatim}
from matplotlib import pyplot
measurements = [3.46, 2.48, 1.98, 3.42, 3.56, 3.87, 3.9, 4.2, 2.9]
data =         [2.44, 1.95, 2.64, 2.84, 1.83, 2.79, 3.5, 3.8, 1.9]
x =            [1.24, 3.15, 4.62, 1.99, 2.01, 4.78, 2.1, 5.7, 3.2]

pyplot.scatter(x,data)
pyplot.scatter(x,measurements)
pyplot.show()

\end{Verbatim}
Notice that the \texttt{pyplot.show()}  is only performed once after
all of the plot objects have been created.

If you want to display multiple plots on the same canvas, you have to
use the \texttt{subplots} command, like this:
\begin{Verbatim}
f, (ax1, ax2) = pyplot.subplots( 2, sharex=True) 
\end{Verbatim}

This command creates a figure with 2 plots on it: one on top of the other.
If you wanted the plots to be in a grid pattern, you could replace the
\texttt{2} in the subplots command with \texttt{(2,2)} (for example).
This would create a 2 x 2 grid of plots, a total of 4 plots.  Notice
that the subplots command is returning 2 things: the figure, and the 2
plots.  The figure is the canvas that the plots are being placed on.
You can create the plot objects like this:
\begin{Verbatim}
from scipy.stats import norm  # Import norm to create some data to plot
f, (ax1, ax2) = pyplot.subplots( 2, sharex=True) 
data= norm.rvs(2.5,0.5,100)  # Create some data to plot
ax1.hist(data,bins = 100)  # Create the first plot
data= norm.rvs(2.5,0.5,1000)  # Create some more data to plot
ax2.hist(data,bins = 100)  # Create the second plot

pyplot.show()  #Show the whole thing.


\end{Verbatim}


