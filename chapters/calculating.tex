\chapter{Calculating}
\label{chap:Calculating}
In a scientific setting, much of what you will ask Python to do will
involve math.  You've already seen how to do very simple math. Here we
will give you all the tools you will need to do any mathematical
calculation you could want.

The calculations that you will perform can be roughly categorized into
two types: (i) simple calculations involving a couple of numbers ( e.g.
$5.5 \sin(6.2 \times \frac{\pi}{2})$ or $5.5 J_0(3.5 \times
\frac{\pi}{4})$) and (ii) more complicated calculations involving many
numbers (e.g. $5.5 \sum_i (x_i - D)^2$ where $x$ is a list of 200,000
numbers)
\section{Simple Calculations}
Any mathematical function that you could possibly need to perform a
simple calculation can be found in the library \code{numpy} or
\code{scipy}. Here is an example:
\begin{Verbatim}
from numpy import sin, pi
from scipy.special import j0 # Bessel function of the 1st kind of
                             # order 0

a = 5.5
b = 6.2
c = pi/2
d = a * sin(b * c)
e = a * j0(3.5 * pi/4)
\end{Verbatim}

Most likely, \code{Numpy} has the mathematical function that you need
and if it doesn't, then \code{scipy.special} probably has it. Tables
\ref{tab:Numpy} and \ref{tab:Scipy} provide a small sampling of the
functions available.  See online help for a more extensive listing.

\section{More Complex Calculations: Don't Use Lists}
What about more complicated calculations involving large data sets?
It may {\em seem} natural to use a list to perform this type of calculation.
However, lists were not designed to do math. For example,
suppose you defined the following list:
\begin{Verbatim}
a = [5.1 , 3.2 , 6.8]
\end{Verbatim}
conceptualizing it as a mathematical vector, and then tried to calculate:
\begin{Verbatim}
b = 3 * a
\end{Verbatim}
expecting \code{b} to become \code{[15.3,9.6,20.4]}.  Try it.  Can you
explain the result? Or maybe you define two lists and try to add them
up, like this:
\begin{Verbatim}
a = [5.1 , 3.2 , 6.8 , 9.2]
b = [2.7 , 1.9 , 3.2 , 9.9]
c = a + b
\end{Verbatim}
Can you explain the results?  So lesson \# 1 is: \ul{Lists} are not
mathematical vectors or matrices and doing vector or matrix math
\underline{cannot be done with a list}.  If you were allowed to do
that, what should Python do when your lists were filled with
non-numerical data, like this:
\begin{Verbatim}
a = ['Physics', 'is']
b = ['the', 'coolest', 'topic']
c = a + b
\end{Verbatim}
  This doesn't mean that the numbers \ul{stored in
  lists} can't be extracted and used in mathematical calculations,
like this
\begin{Verbatim}
a = [4.5,8,2.1,10.8,12]
c = a[0]**a[1]  # Take the first element of a and raise it to a power
                # equal to the second element of a
\end{Verbatim}
it just means that mathematical calculations involving entire arrays
of numbers cannot be done using the list data type.

Second, it may be tempting to associate a
2-D list with a matrix and try to use the \code{:} to slice out a
submatrix.  For example, what if you defined the following 2D list
\begin{Verbatim}
from numpy import array
a = [[1,2,3],[4,5,6],[7,8,9]]
\end{Verbatim}
and interpreted it as this matrix:
\begin{equation}
\left( \begin{tabular}{ccc}
1 & 2 & 3\\
4 & 5 & 6\\
7 & 8 & 9\\
\end{tabular}
\right)
\end{equation}
If you wanted to slice out the following 2 x 2 sub-matrix:
\begin{equation}
\left(\begin{tabular}{cc}
5 & 6 \\
8 & 9 \\
\end{tabular}
\right)
\end{equation}
and you tried to do it like this:
\begin{Verbatim}
a[1:3][1:3]
\end{Verbatim}
you would be disappointed to find out that it did not work that way.
You should take a few minutes to understand what it did do.  In the
end you should always remember to \ul{not treat 2D lists as matrices}.

However, don't think for one second that Python is unable
to handle this kind of math.  Just keep reading and you'll learn how
this is done using the Numpy library. (pronounced num-pie, short for numerical
Python)


\section{Numpy Arrays}
If lists are not designed for mathematical calculations, what should
be used to do this sort of thing?  The answer is a Numpy array. Let's
explore this a little more carefully using a specific example that is
designed to help you see what we mean.  Let's say
you have a large data set
\begin{Verbatim}
x = [2.42762254  2.53691271  3.15932278  1.7128872   2.54105921  2.54094893
  2.55284336  2.36430906  2.37972415  2.70342833  2.2846214   2.37636944
  2.74236195  3.06429336  2.29889954  1.99944808  2.46066766  1.86346638
  2.69619554  1.81298331  2.96144256  3.020208    2.71914935  2.59783385
  2.41512769  2.84674515  2.92394769  3.15879826  2.25886137  3.04074924
  3.14635756  2.60488105  2.79643916  2.67695452  2.77874282  1.94903284
  2.60399377  1.88255081  2.38624122  3.43726289  2.46514806  2.74985076
  2.33684695  2.58710514  2.10996793  3.19191947  3.93418676  2.90987071
  2.52449511  1.71514896  2.42465365  2.24485334  2.88390193  2.97911184
  2.86770773  2.97543667  2.00454583  2.56522443  2.99691011  2.79259592
  2.01617544  1.66098216  2.59230004  2.31295971  3.49570792  2.37890997
  2.14965171  2.40578128  2.44831872  2.0519382   2.41011389  3.07252157
  2.50662296  2.49878442  1.97225157  2.00764702  2.67472532  3.02465629
  2.45257132  2.9325564   2.69301075  2.81356219  2.49886432  1.97998459
  2.86166356  3.24091275  2.83846089  2.58103089  2.23525104  2.85815534
  3.33391592  2.6850452   2.3267767   3.27800198  2.17433118  2.17612604
  2.80002452  2.48975877  3.01856681  2.34280246]
\end{Verbatim}
and you want to calculate the summation
\begin{equation}\label{eq:sum}
\sum_{i=1}^N (x_i - D)^3
\end{equation}
where $x_i$ are the data given above and $D = 5$.  You could
calculate, one-by-one, each contribution in the sum and then add them
up.  However, there is an easier way involving
a library called \code{numpy} .  The main object used in this library
is called an \code{array}, which is very similar to a list except that
an array is intended for mathematical use.  If \code{x} were a Numpy
array, the code to evaluate the sum in equation \ref{eq:sum} would be
as simple as:
\begin{Verbatim}
D=5 #define D so the next line looks exactly like the equation
my_sum=sum((x-D)**3)
\end{Verbatim}

Let's explore arrays a little more.

\subsection*{Array Creation}
There are several ways to create an array.  If you already have a list
of numbers and you just want to convert it to an array, you can do it
with \code{numpy}'s \code{array} function:
\begin{Verbatim}
from numpy import array
xList = [2 , 3 , 5.2 , 2 , 6.7]
xArray = array(xList)
\end{Verbatim}
If you are looking for a function to create an array from scratch,
there are plenty of options.  The function \code{arange} is very similar to
the native \code{range} function that you have already seen.  The
difference is that \code{arange} creates an array object instead of
a list object and \code{arange} allows the stepsize to be less than
one.  Here is an example:
\begin{Verbatim}
from numpy import arange
myArray = arange(0,10,.1)
\end{Verbatim}
This will create an array that looks like this:
\begin{Verbatim}
[0,.1,.2,.3,.4,.5,.6.... 9.5,9.6,9.7,9.8,9.9]
\end{Verbatim}
\marginpar{\footnotesize\captionsetup{type=table} \vspace{-1.5in}
\begin{tabular}{lp{1.05in}}
\code{logspace}        & Returns numbers evenly spaced on a
log scale.  Same arguments as \texttt{linspace}\\ \\
\code{empty}        & Returns an empty array with the
specified shape\\ \\
\code{zeros}        & Returns an array of zeros with the
specified shape\\ \\
\code{ones}        & Returns an array of ones with the
specified shape.\\ \\
\code{zeros\_like}        & Returns an array of zeros with the
same shape as the provided array. \\ \\
\code{fromfile}        & Read in a file and create an array from the
data.\\ \\
\code{copy}        & Make a copy of another array.\\ \\
\end{tabular}
\captionof{table}{A sampling of array-building functions in
  numpy. The arguments to the functions has been omitted to maintain
  brevity.  See online documentation for further details.\label{tab:Arraybuilding}}
}
 Another very useful function for array-creation is \code{linspace},
which creates an array by specifying the starting value, ending value,
and the number of elements that the array should contain.  For example:
\begin{Verbatim}
from numpy import linspace
myArray = linspace(0,10,10)
\end{Verbatim}
This will create an array that looks like this:
\begin{Verbatim}
[0,1.11111,2.222222,3.333333,4.4444444,5.55555555,6.666666,
            7.7777777,8.8888888,10]
\end{Verbatim}
When you use the \code{linspace} function, you aren't specifying a
step size: the distance between values in the array.  You can ask the
\code{linspace} command to tell you what step size results from the
array that it created by adding the \code{retstep =True} as an
argument to the function,like this:
\begin{Verbatim}
from numpy import linspace
myArray,mydx = linspace(0,10,10,retstep = True)
\end{Verbatim}
Notice that since \code{linspace} is now returning two things(the
array and the step size), we must have two variables to assign them to
on the left hand side of the equals sign.

Many other useful function for creating arrays are available.  Online
documentation is freely available.  Table \ref{tab:Arraybuilding}
gives some of the more heavily-used ones:

\subsection*{Simple Math with Arrays}
Once the array object is created, a whole host of mathematical
operations become available.  For example, you can square the array
and Python knows that you want to square each element, or you can add
two arrays together and Python knows that you want to add the
individual elements of the arrays.  You can add a constant value to
every element of an array, or even multiply two arrays together and
the elements of the first array are multiplied by the corresponding
element in the second.  Here's a sampling of examples.
\begin{Verbatim}
from numpy import array
xList = [2 , 3 , 5.2 , 2 , 6.7]
xArray = array(xList)    # Create first array
yArray = array([4,8,9.8,2.1,8.2,4.5])  # Create second array

c = xArray**2   # Square the elements of the first array
d = xArray + 3  # Add 3 to every element of the first array
e = xArray * 5  # Multipy every element of the first array by 5
f = xArray + yArray  # Add the elements of array one to the elements of
                     # array two
g = xArray * yArray  # Multiply the elements of array one by
                     # the elements of array two

\end{Verbatim}
In short, you can do all of the math that you were hoping you could do
when you first learned about Python lists.

\subsection*{Mathematical functions with Arrays}
\marginpar{\footnotesize\captionsetup{type=table}
  \vspace{0.5in}
\begin{tabular}{l}
\texttt{sin(x)} \\
\texttt{cos(x)} \\
\texttt{tan(x)} \\
\texttt{arcsin(x)} \\
\texttt{arccos(x)} \\
\texttt{arctan(x)} \\
\texttt{sinh(x)} \\
\texttt{cosh(x)} \\
\texttt{tanh(x)} \\
\texttt{sign(x)} \\
\texttt{exp(x)} \\
\texttt{sqrt(x)} \\
\texttt{log(x)} \\
\texttt{log10(x)} \\
\texttt{log2(x)} \\
\end{tabular}
\captionof{table}{A very small sampling of functions belonging to the
  \texttt{numpy} library.\label{tab:Numpy}}
}

Mathematical functions (like $\sin(x)$, $\sinh(x)$...etc.) can be
evaluated using arrays as their arguments and the result is an array
of function values.  (Just imagine wanting to calculate the $\sin$ of
1,000,000 numbers!!) However, you must use functions from either the
\code{numpy} or \code{scipy} library, and not from the \code{math}
library.  The functions from \code{numpy} and \code{scipy} library are
designed to work on arrays but the one from the \code{math} library is
not. Here is an example
\begin{Verbatim}
import numpy as np
import math

xList = [2 , 3 , 5.2 , 2 , 6.7]
xArray = array(xList)

c = np.sin(xArray)   # Works just fine, returning array of numbers.
d = math.sin(xArray) # Returns an error.
\end{Verbatim}
Notice that \code{math}'s version of \code{sin} does not
know what to do when you give it an array of numbers.  It only works
for single numbers.  \code{Numpy}'s \code{sin} function, on the
other hand, does know what to do with an array of numbers: it
calculates the sin of all the numbers.


\section{Accessing and Slicing Arrays}
Accessing and slicing arrays can be done in exactly the same way as is
done with lists.  However, there is some additional functionality for
accessing and slicing arrays that do not apply to lists.
\subsection*{Accessing Multiple Elements}
Let's consider the following 1-d array:
\begin{Verbatim}
from numpy import array
a = array([1,2,3,4,5,6,7,8,9,10])
\end{Verbatim}
Let's say you wanted to extract elements 2,3,5,and 9.  You can extract
all of these elements in one shot by using a list of these numbers for
the index, like this:
\begin{Verbatim}
b = a[[2,3,5,9]]
\end{Verbatim}
You can't do that with lists.

When using two-dimensional arrays, like this one:
\begin{Verbatim}
from numpy import array
a = array([[1,2,3],[4,5,6],[7,8,9]])
\end{Verbatim}
single elements of this array can easily be extracted, just as with
lists:
\begin{Verbatim}
from numpy import array
a = array([[1,2,3],[4,5,6],[7,8,9]])
b = a[0][2]
\end{Verbatim}
This will extract the number 3, which is the third element in the
first list.  In other words, it's element 0,2 in array \code{a}.
What if you wanted to extract multiple elements in one shot.  Could
you do that?  In fact you can and here is how:
\begin{Verbatim}
from numpy import array
a = array([[1,2,3],[4,5,6],[7,8,9]])
b = a[[0,2,1],[1,1,2]]
\end{Verbatim}
Pay special attention to the last line.  It will extract the numbers
2,8, and 6.  The first list (\texttt{[0,2,1]}) indicate the rows
(first dimension of the array) where the element is to be extracted
and the second list (\texttt{[1,1,2]}) indicate the columns that
correspond to the rows provided.  So elements \texttt{[0,1]},
\texttt{[2,1]}, and \texttt{[1,2]} from list \texttt{a} will be
extracted.  The lists of indices can be as long as you want them to be
and all of the corresponding elements will be extracted.
\subsection*{Boolean Slicing}
Imagine wanting to extract the elements of an array that meet some
provided criteria.  An array can be sliced according to the provided
criteria very easily by replacing the index with the selection
criteria as a boolean statement.
\begin{Verbatim}
from numpy import array
a = array([1,2,3,4,5,6])
b = a[a>2]
\end{Verbatim}
Here, only those elements that are greater than 2 will be extracted
from the array.  It will work on multi-dimensional arrays as well

\subsection*{Multi-dimensional Slicing}
We've already shown you how to slice a list using the \code{:}
operator.  The same can be done with arrays.  However, for 2D (and
higher) arrays the slicing is more powerful (intuitive).  It can be
helpful to visualize an array as a matrix, even if it is not being
treated that way Mathematically.  For example, let's say that you
define the following array:
\begin{Verbatim}
from numpy import array
a = array([[1,2,3],[4,5,6],[7,8,9]])
\end{Verbatim}
which could be interpreted as this matrix:
\begin{equation}
\left( \begin{tabular}{ccc}
1 & 2 & 3\\
4 & 5 & 6\\
7 & 8 & 9\\
\end{tabular}
\right)
\end{equation}
If you wanted to slice out the following 2 x 2 sub-matrix:
\begin{equation}
\left(\begin{tabular}{cc}
5 & 6 \\
8 & 8 \\
\end{tabular}
\right)
\end{equation}
you could do it like this:
\begin{Verbatim}
b[1:3,1:3]  # Slice out a sub-array (Exactly what you wanted!)
\end{Verbatim}
If you want all of the elements in a given dimension, use the \code{:}
alone with no numbers surrounding it. For example, the following:
\begin{Verbatim}
b[:,1:3]
\end{Verbatim}
would extract all of the rows on columns 1 and 2:
\begin{equation}
\left(\begin{tabular}{cc}
2 & 3 \\
5 & 6 \\
8 & 8 \\
\end{tabular}
\right)
\end{equation}
\marginpar{\footnotesize\captionsetup{type=table}
  \vspace{-1.0in}
\begin{tabular}{lp{1.00in}}
\texttt{airy(z)}& Airy function  \\
\texttt{jv(z)} & Bessel function of 1st kind\\
\texttt{yv(z)} & Bessel function of 2nd kind\\
\texttt{kv(n,z)} & Modified Bessel function of 2nd kind of integer
order n\\
\texttt{iv(v,z)} & Modified Bessel function of 1st kind of real
order v\\
\texttt{hankel1(v,z)} & Hankel function of 1st kind of real
order v\\
\texttt{hankel2(v,z)} & Hankel function of 2nd kind of real
order v\\
\end{tabular}
\captionof{table}{A very small sampling of functions belonging to the
  \texttt{scipy.special} library.\label{tab:Scipy}}
}
This kind of slicing just can't be done with lists.  Also note that
you must use the \code{[x1:y1,x2:y2]} notation rather than the
\code{[x1:y1][x2:y2]} notation, familiar with lists.  Use of the
latter will not fail, but it will not produce the sub-matrix desired.


\section{Miscellaneous Mathematical Functions}

\subsection*{Summing the elements}
Suppose you have a list(or array) of numbers and you'd like to add up
all of the elements.  Python has a built-in \code{sum} function and
there is also a \code{sum} function inside of \code{numpy} that
will do this. They both do the same thing for one-dimensional lists.
\begin{Verbatim}
a = [1.5,2.2,9.8,4.6]
b = sum(a)   #Use built-in sum function
from numpy import sum
c = sum(a)   # Use numpy's sum function
\end{Verbatim}
If you are summing up the elements of a two-dimensional list, the
built-in version of \code{sum} will not work and you will have to
use \code{numpy}'s version.
\begin{Verbatim}
a = [[1.5,2.2],[9.8,4.6]]  # Define a 2-d list, a list of lists
b = sum(a)   #Use built-in sum function, notice the error
from numpy import sum
c = sum(a)   # Use numpy's sum function, no error.
d = sum(a,axis = 1)
\end{Verbatim}
Notice the extra argument to the \code{sum} function in the last
line.  The \code{axis=1}\sidenote{By default, \code{axis=0}.  Therefore \code{sum(a)} and \code{sum(a,axis=0)} perform the same operation.} indicates that you want to sum up the
elements in each individual list and return a list of sums.  If you
had a higher-dimensional list, you could use \code{axis=2} or
\code{axis = 3} as well.


\marginpar{\footnotesize\captionsetup{type=table} \vspace{0in}
\begin{tabular}{lp{1.05in}}
\code{sum}& sum up the elements of an array.\\
\code{prod}& find the product of the elements of an array.\\
\code{cumsum}& cumulative sum up the elements of an array.\\
\code{cumprod}& cumulative product up the elements of an array.\\
\code{diff}        & discrete difference between all adjacent elements.\\
\code{cross}        & cross product of two vectors.\\
\code{inner}        & dot(inner) product of two vectors.\\

\end{tabular}
\captionof{table}{A sampling of miscellaneous functions in
  numpy.  See online documentation for further details.\label{tab:NumpyMisc}}
}
\section{Complex Arithmetic}
Python can work with complex numbers just as easily as real numbers.
The variable \texttt{j} is the imaginary number $\sqrt{-1}$ and a
complex number can be specified by placing the \texttt{j} after the
imaginary number:
\begin{Verbatim}
a = 1 + 2j
b = 2 + 3j
print(a+b)
\end{Verbatim}
You may also create a complex number using the \texttt{complex}
function:
\begin{Verbatim}
a = complex(1,2j)
b = complex(2,3j)
print(a+b)
\end{Verbatim}
There are several native functions that work on complex numbers:
\begin{Verbatim}
a = 1 + 2j
b = 2 + 3j
c = a.real
d = a.imag
e = a.conjugate()
\end{Verbatim}
More functions are available in the \texttt{cmath} package.
