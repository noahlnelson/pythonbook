\chapter{Functions and Libraries}
\label{chap:Functions}

In the previous chapter I showed you how to perform very simple
mathematical calculations using variables.  You were also briefly
introduced to some simple functions that are available in the Python
language.  However, you were probably left wondering about more complex
mathematical calculations, like $sin(\frac{\pi}{2})$ or $e^{2.5}$.
The native Python language does not include functions for these
mathematical operations and they must be imported from libaries.  In
this chapter we will discussion functions and libraries.

\section{User-defined functions}
A function is simply a set of instructions to be performed when called
upon.  You (the user) can create your own function like this
\begin{Verbatim}
def myFunction(a,b):
    c = a + b
    d = 3.0 * c
    f = 5.0 * d**4
    return f
\end{Verbatim}
This function performs several simple calculations and then uses the
\texttt{return} statement to pass the final result back out of the
function.  Every user-defined function must begin with the keyword
\texttt{def} followed by the function name (you can choose it). The
variables \texttt{a} and \texttt{b} are the arguments to the function.
This means that when you want to call this function, you must provide
these numbers.  This function can be called like this
\begin{Verbatim}
r = 10
t = 15
result = myFunction(r,t)
\end{Verbatim}
In this case, when the function is called, \texttt{a} gets assigned
the value of $10$ and \texttt{b} gets assigned the value of $15$.  The
result of this calculation is stored in the variable \texttt{result},
which is outside the function.  You can think of a function as a black
box. You create the contents of the box and specify what information
must be passed into the black box.  The user of the black box need not
know, or fully understand, everything that the designer put inside
it.  He only needs to know what to give the box and what the box will
give back to him.

\section{Native functions}
Python has many native functions: functions that are included with the
programming language.  You have already been using some of them, like these
\begin{Verbatim}
len(mylist)  % Returns the length of a list.
float(5)     % Converts an integer to a float.
str(67.3)    % Converts a float to a string.
\end{Verbatim}
The functions: \texttt{len}, \texttt{float}, and \texttt{str} are all
built-in functions, and they each take a single argument.  Other
built-in function are found in the margin tables in the previous chapter.

\section{Imported Functions and Libraries}
If a function isn't included in the standard Python distribution and
you don't want to write your own function, you may be able to find the
function that you need elsehwere.  For the vast majority of the things
you will want to do, there will be an existing libary that has the
function you need. Many third-party developers have
written very useful functions and bundled them into libraries.  To
gain access to these great tools requires that you first install
them.\sidenote{If you are using Enthought's Canopy package, all of the
  libraries that you need are already installed} Once they are
installed, you can import the library of functions like this:
\begin{Verbatim}
import math
\end{Verbatim}
This imports a library called \texttt{math}.  You can use a function
inside of this libary like this
\begin{Verbatim}
math.sqrt(5.2)  % Take the square root of 5.2
math.pi         % Get the value of pi
math.sin(34)    % Find the sine of 34 radians
\end{Verbatim}
Using the functions inside a libary requires that you know what
functions are available.  This information is usually available in the
library's documentation.  Google will be a great resource here.

A library can be imported and then called by a different name like
this:
\begin{Verbatim}
import math as mt
\end{Verbatim}
Here, the short name \texttt{mt} was chosen for this library.
The desired functions can then be called like this
\begin{Verbatim}
mt.sqrt(5.2)  % Take the square root of 5.2
mt.pi         % Get the value of pi
mt.sin(34)    % Find the sine of 34 radians
\end{Verbatim}

Sometimes you may not want to import the entire library, just a few
functions. This can be done like this
\begin{Verbatim}
from math import sqrt
\end{Verbatim}
and the \texttt{sqrt} function can then be used without the library
name before it, like this
\begin{Verbatim}
sqrt(5.5)
\end{Verbatim}
If you want to import every function inside of a libary, do this
\begin{Verbatim}
from math import *
\end{Verbatim}
Now, every function contained in \texttt{math} is available without
needing the \texttt{math.} prefix in front of it.

Throughout this book, we will use a variety of libaries to accomplish
important tasks.  Instead of giving a complete description of each
library used and every function that it contains, we will simply
discuss the functions needed for each specific task.  The interested
reader is referred to the documentation of the various libaries for
further details.

\section{The numpy (numerical python) library}
Previously, we learned about lists.  Recall that lists were not
designed to do vector/matrix math.  For all of your math needs in
python, you should use a library called Numpy\sidenote{Short for
  numerical python} (prounounced num-pie).  Covering every function
avaialble in the numpy library would take way too long.  For now,
we'll just discuss some common tasks.

\subsection*{Arrays}
Arrays are numpy objects and are the main object used by this
library.  You can create an array using numpy's array function, like
this
\begin{Verbatim}
from numpy import array

a = [1,5,6,7]
b = array(a)     
\end{Verbatim}
The \texttt{array} function converts a python list to a numpy array.
There are many other ways to create numpy arrays.  Here are a few
examples
\begin{Verbatim}
from numpy import zeros, eye

a = zeros(5)
b = zeros([5,5])
b = eye(5)       
\end{Verbatim}

\texttt{zeros} will create an array of zeros, with the size of the
array specified as an argument.



