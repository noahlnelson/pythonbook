\chapter{Statistics and Random Number Generation}

\label{chap:Statistics}
\section{Simple Statistical Functions}
Doing science often involves collecting large amounts of data and then
analyzing the data to extract conclusions.  You may find yourself
wanting to calculate the mean, standard deviation, and/or other
statistical functions.  Numpy can do that for you (here I've assumed
that \code{x} contains the data set):
\begin{Verbatim}
from numpy import mean, std

avg = mean(x)
standardDev = std(x)
\end{Verbatim}
Other useful statistical functions can be found in table \ref{tab:numpystats}

\marginpar{\footnotesize\captionsetup{type=table}
  \vspace{0.5in}
\begin{tabular}{lp{0.8in}}
\code{median}&  Median of data\\
\code{average}&  Compute weighted average\\
\code{mean}&  Compute the arithmetic mean\\
\code{nanmean}&  Compute the arithmetic mean, ommiting nan(not a number) entries\\
\code{percentile}&  Compute a given percentile of the data\\
\code{std}&  Standard deviation\\
\code{histogram}&  Compute histogram of data.\\
\end{tabular}
\captionof{table}{A very small sampling of functions belonging to the
  \texttt{numpy} library.\label{tab:numpystats}}
}

Let me explain the function \code{nanmean} briefly.  NaN is shorthand
for ``not a number'' and this ``value'' appears when an illegal math
operation is attempted. The ``value'' \code{inf} may also appears when
  mathematical operations are attempted that lead to a very large
  result.  For example, try the following:
\begin{Verbatim}
from numpy import array,sqrt
a = array([[-2.,0.,1.,2.,3.,4.]])
print sqrt(a)
print a/0
\end{Verbatim}
and look at the results.  You should see arrays with non-numerical
entries (\code{nan} or \code{inf}).  If, in the process of
post-processing your data, you end up with a \code{nan} or \code{inf}
buried deep in your data set, the function \code{mean} will not work correctly.
It would be annoying if you had to sift through your
data manually and delete all of the \code{nan}s before you could
calculate the mean.  Numpy thought of this.  The function
\code{nanmean} does the same thing that \code{mean} does except that it
ignores any entries that are nans.  Many of the other
standard functions have ``nan'' counterparts too.


\section{Random Number Generation}
Often while programming, you'll need to generate a random number over
some interval.  What you actually mean when you ask for a random
number is to sample from a statistical distribution.  For example, the
Gaussian (Normal) distribution is peaked in the center and drops off
to zero on both sides of the peak.  If I were to sample from a Gaussian
distribution, numbers close to the peak are more likely to be chosen
over numbers that are close to the tails of the distribution.  There
are many statistical distributions, each with their place in science,
and you should always think about which distribution is appropriate
for your problem.  

The simplest distribution is the uniform (or flat) distribution and
numpy has a sub-libary that can sample from it (and many others)

\begin{Verbatim}
from numpy.random import uniform

a = uniform(2,5,1000)
\end{Verbatim}

This will generate an array of 1000 random samples from the uniform
distrubution in the range $(2,5)$

Samples from a Gaussian distribution can be generated too

\begin{Verbatim}
from numpy.random import normal

a = normal(5,0.5,1000)
\end{Verbatim}

This will generate 1000 random samples from a Gaussian distribution
centered at $5$(location of peak) with a variance of $0.5$ (width of peak)
A small sampling of the statistical distributions available via numpy
are given in table \ref{tab:distributions}.  For all of the
distributions, the optional argument \code{size=} can be added to
specify the number of samples you'd like.

\section{Plotting}
Statistical plotting tool range from plotting a continuous
distribution to making a histogram of some collected data.  Plotting a
continuous distribution is nothing more than plotting the function
corresponding to that distribution.  However, instead of having to
look up that function, \code{scipy.stats} makes your life a little easier by
making the functions available to you.
\marginpar{\footnotesize\captionsetup{type=table}
  \vspace{-4.5in}
\begin{tabular}{lp{0.95in}}
\code{rvs(size=100)}  & Sample from the distribution. size defaults
to 1 if omitted. \\
\code{pdf(x)}  & Probability density function. \\
\code{cdf(x)}  & Cumulative density function. \\
\code{logpdf(x)}  & Log of the probability density function. \\
\code{logcdf(x)}  & Log of the cumulative density function. \\
\code{stats()}  & Returns the mean,variance,skew, and/or kurtosis of
the distribution. \\
\end{tabular}
\captionof{table}{A sampling of functions available to any statistical
  distribution inside of \code{scipy.stats}.\label{tab:distfunctions}}
}

\subsection*{Continuous Distributions}
While \code{numpy} has functions for performing simple statistical
operations and sampling a distribution, \code{scipy.stats} has some
additional functionaliy.  One extra feature contained in this library
is the ability to plot the distribution.  Here is an example:


\begin{Verbatim}
from scipy.stats import norm #import the distribution you want.
from numpy import linspace
from matplotlib import pyplot

x = linspace(-4,4,100)
dist = norm.pdf(x)
pyplot.plot(x,dist,lw=5,alpha=0.6)
pyplot.show()
\end{Verbatim}

Notice the use of the \code{pdf} (short for probability density
function) function that is associated with the distribution
\code{norm}.  To fully understand what is happening here requires that
you understand classes.  When you imported \code{norm} from
\code{scipy.stats}, you actually didn't import a function like you
thought.  Instead, you imported an ``object'', and associated with
that object are a long list of variables and functions.
\marginpar{\footnotesize\captionsetup{type=table}
  \vspace{-4.5in}
\begin{tabular}{lp{1.05in}}
\code{beta(a,b)}  & Sample from a beta(a,b) \\
\code{chisquared(a)}  & Sample from a $\chi^2(a)$ \\
\code{uniform(a,b)}  & Sample from a uniform in range $a < x < b$ \\
\code{normal(a,b)}  & Sample from a normal distribution \\
\code{poisson(a)}  & Sample from a poisson distribution \\
\end{tabular}
\captionof{table}{A sampling of statistical distributions available in
  \code{numpy.random}.\label{tab:distributions}}
}  One of those
functions is \code{pdf}.  Other functions that are available for
distributions from \code{scipy.stats} are givein in table
\ref{tab:distfunctions}.


\subsection*{Histograms}

A histogram is generated by binning your domain and then counting up
the number of data points in each bin.  A histogram can be generated
as follows
\begin{Verbatim}
from numpy.random import normal
from matplotlib import pyplot

a = normal(3,0.5,1000)
pyplot.hist(a)
pyplot.show()
\end{Verbatim}
 Here we have generated some data from a Gaussian distribution and then
plotted a histogram of the samples.





