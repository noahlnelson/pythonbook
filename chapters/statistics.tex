\chapter{Statistics and Random Number Generation}

\label{chap:Statistics}
\section{Simple Statistical Functions}

\section{Random Number Generation}
Often while programming, you'll need to generate a random number over
some interval.  What you actually mean when you ask for a random
number is to sample a statistical distribution.  For example, the
Gaussian (Normal) distribution is peaked in the center and drops off
to zero on both sides of the peak.  If I were to sample a Gaussian
distribution, numbers close to the peak are more likely to be chosen
over numbers that are close to the tails of the distribution.  There
are many statistical distributions, each with their place in science,
and you should always think about which distribution is appropriate
for your problem.  

The simplest distribution is the uniform (or flat) distribution and
numpy has a sub-libary that can sample from it (and many others)

\begin{Verbatim}
from numpy.random import uniform

a = uniform(2,5,1000)
\end{Verbatim}

This will generate an array of 1000 random samples from the uniform
distrubution in the range $(2,5)$

Samples from a Gaussian distribution can be generated too

\begin{Verbatim}
from numpy.random import normal

a = normal(5,0.5,1000)
\end{Verbatim}

This will generate 1000 random samples from a Gaussian distribution
centered at $5$(location of peak) with a variance of $0.5$ (width of peak)


\section{Plotting}

\subsection*{Continuous Distributions}


\subsection*{Histograms}

A histogram is generated by binning your domain and then counting up
the number of data points in each bin.  A histogram can be generated
as follows
\begin{Verbatim}
from numpy.random import normal
from matplotlib import pyplot

a = normal(3,0.5,1000)
pyplot.hist(a)
pyplot.show()
\end{Verbatim}
Here we have generated some data from a Gaussian distribution and then
plotted a histogram of the samples.


\marginpar{\footnotesize\captionsetup{type=table}
  \vspace{-4.5in}
\begin{tabular}{lp{1.05in}}
\code{sum(a)}  & Returns the sum of the elements of \code{a} \\ \\
\end{tabular}
\captionof{table}{A sampling of ``housekeeping'' functions for
lists.\label{tab:HousekeepingList}}
}


